% Destacar o sumário da presente seção antes de iniciá-la

\AtBeginSection[]{
	\begin{frame}
		\frametitle{}
		\tableofcontents[currentsection]
	\end{frame}
}

%%%%%%%%%%%%%%%%%%%%%%%%%%%%%%%%%%%%

	\section{Contextualização do Tema}
	\begin{frame}{Contextualização do Tema} % Página 1
		\begin{block}<+->{Contexto Histórico}
			\justify
			“Pelo fim do século dezenove, a ênfase no rigor levou numerosos matemáticos à produção de exemplos de funções ’patológicas’ que, devido a alguma propriedade incomum, violavam um teorema que antes se supunha válido em geral” (BOYER, 2012, p.415).
		\end{block}
	
	% Se não quiser que as coisas apareçam aos poucos, remova [<+->] das listas e <+-> dos blocos
	
	%Modelo de lista de itens
	%\begin{itemize}[<+->]
	%	\item Item 1
	%	\item Item 2
	%	\item Item 3
	%\end{itemize}
	
	
	\end{frame} % Fim da página 1
	
	\begin{frame}{Contextualização do Tema} % Página 2
	% Modelo de blocos
		\begin{block}<+->{Percepção de Lebesgue}
			\justify
			Lebesgue, refletindo sobre o trabalho de Borel sobre conjuntos, viu que a definição de Riemann de integral tem o defeito de só se aplicar a casos
			excepcionais, pois assume não mais que uns poucos pontos de descontinuidade para a função.
			Se uma função $y = f(x)$ tem muitos pontos de descontinuidade, então, à medida que o intervalo $x_{i+1} - x_i$
			se torna menor, os valores $f(x_{i+1})$ e $f(x_i)$ não ficam
			necessariamente próximos (BOYER, 2012, p.416).
		\end{block}
	\end{frame} % Fim da página 2


	\begin{frame}{Contextualização do Tema} % Página 3
		% Modelo de blocos
		\begin{block}<+->{Solução do Problema}
			\justify
			Em vez de subdividir o domínio da variável independente, Lebesgue dividiu, portanto, o campo de variação $\overline{f} - f$ 
			da função em subintervalos $\Delta y_i$ e em cada subintervalo
			escolheu um valor $\eta_1$. 
			Então, achou a 'medida' $\mu(E_i)$ do conjunto $E_i$ dos pontos do eixo $x$ para os
			quais os valores de $f$ são aproximadamente iguais a $\eta_1$ (BOYER, 2012, p.416).
		\end{block}
	\end{frame} % Fim da página 3

	\begin{frame}{Contextualização do Tema} % Página 5
		% Modelo de blocos
		\begin{block}<+->{Pesquisa Prévia}
			\justify
			\begin{itemize}[<+->]
				\item Pesquisa com a frase “ensino de cálculo diferencial e integral” na
				Biblioteca Digital Brasileira de Teses e Dissertações (BDTD), sem aspas;
				\item Retorno de 310 trabalhos em 29/10/2023;
				\item Observação de Título e Resumo;
				\item Não constava o ensino da integral de Lebesgue para alunos de graduação em Ciências Exatas.
			\end{itemize}
		\end{block}
	\end{frame} % Fim da página 5
	
	\begin{frame}{Pergunta Diretriz} % Página 6
		% Modelo de blocos
		\begin{block}{}
			\justify
			Mediante esta ausência, esta pesquisa levanta o seguinte questionamento: de que
			forma pode ser apresentada a teoria da medida e integração de Lebesgue de maneira elementar
			para os alunos de graduação em ciências exatas?
		\end{block}
	\end{frame} % Fim da página 6

	\begin{frame}{Objetivos} % Página 7
		% Modelo de blocos
		\begin{block}<+->{Objetivo Geral}
			\justify
			Conhecer algumas definições e resultados da teoria da medida e da integração de Lebesgue de forma elementar.
		\end{block}
		\begin{block}<+->{Objetivos Específicos}
			\justify
			 \begin{itemize}[<+->]
				\item Definir a base do estudo da teoria da medida por meio dos espaços mensuráveis;
				\item Conhecer a teoria da medida de maneira generalizada;
				\item Descrever o processo da construção da integral de Lebesgue mediante o avanço da teoria da medida.
			\end{itemize}
		\end{block}
	\end{frame} % Fim da página 7

	\begin{frame}{Metodologia} % Página 7
	% Modelo de blocos
		\begin{block}<+->{Elementos da pesquisa}
			\justify
		 	\begin{itemize}[<+->]
				\item Natureza Básica;
				\item Carácter Exploratório;
				\item Procedimento Técnico de Revisão Bibliográfica.
			\end{itemize}
		\end{block}
	\end{frame} % Fim da página 7













