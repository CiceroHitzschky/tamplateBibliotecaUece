The theory of measure and integration is a crucial topic for advancing studies in mathematics. Initially developed by Bernhard Riemann (1826-1866), Georg Cantor (1845-1918), and Emile Borel (1871-1956), it was later generalized by Henri Lebesgue (1875-1941). Originally, its development aimed to generalize Riemann's integral, addressing the limitation of only being applicable to exceptional cases with few points of discontinuity. Nowadays, it finds applications in diverse areas such as Functional Analysis, Probability, Statistics, Partial Differential Equations, among others.
%
Despite its contemporary significance, Lebesgue integration theory is not commonly introduced to undergraduate students in exact sciences. This work seeks to provide an introductory overview of the theory of measure and Lebesgue integration for undergraduate students in natural sciences, highlighting some of the most relevant and fundamental results. This is achieved using a basic methodology with a quantitative approach through a literature review.
%
Through this review, we also attain the following specific objectives: define the foundational aspects of measure theory through measurable spaces, comprehend the generalized theory of measure, and elucidate the construction process of Lebesgue's integral through the progression of measure theory.
%
Finally, a proposal is suggested for presenting the topic covered in this work to undergraduate students in exact sciences.

% Keywords separated by a period
\keywords{measure theory; Lebesgue integration theory; calculus teaching.}