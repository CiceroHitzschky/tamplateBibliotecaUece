%%%%%%%%%%%%%%%%%%%%%%%%%%%%%%%%%%%%%%%%%%%%%%%%%%%%%%%%%%%%%%%%%%%%%%%%
%% Customizações do abnTeX2 (http://abnTeX2.googlecode.com)           %%
%% para a Universidade Estadual do Ceara - UECE                       %%
%%                                                                    %%
%% This work may be distributed and/or modified under the             %% 
%% conditions of the LaTeX Project Public License, either version 1.3 %%
%% of this license or (at your option) any later version.             %%
%% The latest version of this license is in                           %%
%%   http://www.latex-project.org/lppl.txt                            %%
%% and version 1.3 or later is part of all distributions of LaTeX     %%
%% version 2005/12/01 or later.                                       %%
%%                                                                    %%
%% This work has the LPPL maintenance status `maintained'.            %%
%%                                                                    %%
%% The Current Maintainer of this work is Thiago Nascimento           %%
%%                                                                    %%
%% Project available on: https://github.com/thiagodnf/uecetex2        %%
%%                                                                    %%
%% Further information about abnTeX2                                  %%
%% are available on http://abntex2.googlecode.com/                    %%
%%                                                                    %%
%%%%%%%%%%%%%%%%%%%%%%%%%%%%%%%%%%%%%%%%%%%%%%%%%%%%%%%%%%%%%%%%%%%%%%%%

% Alteração da ABNT
\bibliographystyle{abntex2-2023.bst}

% Importações de pacotes
\usepackage[utf8]{inputenc}                         % Acentuação direta
\usepackage[T1]{fontenc}                            % Codificação da fonte em 8 bits
\usepackage{graphicx}                               % Inserir figuras
\usepackage{amsfonts, amssymb, amsmath}             % Fonte e símbolos matemáticos

\usepackage{tikz}
\usetikzlibrary{spy}
\usepackage{pgf}
\usepackage{pgfplots}
\usetikzlibrary{positioning,shapes,fit,calc}
\usetikzlibrary{patterns}
\usepackage{ragged2e}
\usepackage{booktabs}                               % Comandos para tabelas
\usepackage{verbatim}                               % Texto é interpretado como escrito no documento
\usepackage{multirow, array}                        % Múltiplas linhas e colunas em tabelas
\usepackage{multicol}
\usepackage{indentfirst}                            % Endenta o primeiro parágrafo de cada seção.
\usepackage{listings}                               % Utilizar codigo fonte no documento
\usepackage{xcolor}
\usepackage{microtype}                              % Para melhorias de justificação?
\usepackage[portuguese,ruled,lined]{algorithm2e}    % Escrever algoritmos
\usepackage{algorithmic}                            % Criar Algoritmos  
%\usepackage{float}                                  % Utilizado para criação de floats
\usepackage{amsgen}
\usepackage{lipsum}                                 % Usar a simulação de texto Lorem Ipsum
%\usepackage{titlesec}                               % Permite alterar os títulos do documento
\usepackage{tocloft}                                % Permite alterar a formatação do Sumário
\usepackage{etoolbox}                               % Usado para alterar a fonte da Section no Sumário
\usepackage[nogroupskip,nonumberlist,acronym]{glossaries}                % Permite fazer o glossario
\usepackage{caption}                                % Altera o comportamento da tag caption
\usepackage[alf, abnt-emphasize=bf, bibjustif, recuo=0cm, abnt-etal-cite=3, abnt-etal-list=0,abnt-etal-text=it]{abntex2cite}  % Citações padrão ABNT
%\usepackage[bottom]{footmisc}                      % Mantém as notas de rodapé sempre na mesma posição
%\usepackage{times}                                 % Usa a fonte Times
\usepackage{mathptmx}                               % Usa a fonte Times New Roman										
%\usepackage{lmodern}                               % Usa a fonte Latin Modern
%\usepackage{subfig}                                % Posicionamento de figuras
%\usepackage{scalefnt}                              % Permite redimensionar tamanho da fonte
%\usepackage{color, colortbl}                       % Comandos de cores
%\usepackage{lscape}                                % Permite páginas em modo "paisagem"
%\usepackage{ae, aecompl}                           % Fontes de alta qualidade
%\usepackage{picinpar}                              % Dispor imagens em parágrafos
%\usepackage{latexsym}                              % Símbolos matemáticos
%\usepackage{upgreek}                               % Fonte letras gregas
\usepackage{appendix}                               % Gerar o apendice no final do documento
\usepackage{paracol}                                % Criar paragrafos sem identacao
\usepackage{lib/uecetex2}		                    % Biblioteca com as normas da UECE para trabalhos academicos
\usepackage{pdfpages}                               % Incluir pdf no documento
\usepackage{amsmath}                                % Usar equacoes matematicas
\usepackage{chngcntr}
\usepackage{colortbl}
\counterwithout{equation}{chapter} 
% \usepackage{setspace} %mudança do tam. da letra??? 
\usepackage{leading}

%%%%%%%%%%%%%%%%%%%
\newtheorem{corollary}{Corolário}[section]
\newtheorem{proposition}{Proposição}[section] 
\newtheorem{definition}{Definição}[section]
\newtheorem{theorem}{Teorema}[section]
\newtheorem{lemma}{Lema}[section]
\newtheorem{problem}{Problema}[section]
\newtheorem{example}{Exemplo}[section]
\newtheorem{counterexample}{Contraexemplo}[section]
\newtheorem{remark}{Observação}[section]
\newtheorem{axiom}{Axioma}
%%%%%%%%%%%%%


% Organiza e gera a lista de abreviaturas, simbolos e glossario
\makeglossaries

% Gera o Indice do documento
\makeindex


%%%%%%%%%%%%%%%%%%%%%%%%%%%%%%% Citações diretas  %%%%%%%%%%%%%%%%%%%%%%%%%% setstretch{1.0} \singlespacing

\newenvironment{citlon}
{\vspace{0.5cm} \hfill\begin{minipage}[c]{12cm}\leading{13pt}\small }
{\end{minipage}\vspace{0.5cm}}
% \hfill preenche o texto a direita

%%%%%%%%%%%%%%%%%%%%%%%%%%%%%%% Minhas Modificações  %%%%%%%%%%%%%%%%%%%%%%%%%% 
\usepackage{csquotes}                                % Permite o uso das aspas por meio do \quote{}

\newcommand{\R}{\mathbb{R}}
%%%%%%%%%%%%%%%%%%%%%%%

\newcommand{\dlim}{\displaystyle \lim}

\newcommand{\dint}{\displaystyle \int}

\newcommand{\dsum}{\displaystyle \sum}

\DeclareMathOperator{\seno}{\textrm{\ seno\ }}
\DeclareMathOperator{\xreta}{\overline{\R}}

\DeclareMathOperator{\menfus}{\textit{M}(\textit{X}, \mathcal{C} )}

\DeclareMathOperator{\xborel}{\overline{\borel}}

\renewcommand{\mod}[3]{#1 \equiv #2 \ (\textrm{mod} \ #3)}

\DeclareMathOperator{\dps}{\displaystyle}
\newcommand{\sigal}{$\sigma$-álgebra\;}
\newcommand{\sigals}{$\sigma$-álgebras\;}
\DeclareMathOperator{\borel}{\mathcal{B}}
\DeclareMathOperator{\cc}{\mathcal{C}}
\DeclareMathOperator{\pp}{\mathcal{P}}


\newcommand{\supercite}[2]{(\citeauthor{#1}, \citeyear{#1}, #2)}
%%%%%%%%%%%%%%%%%%%%%%%%%%%%%%%%%%%%%%%%%%

\DeclareMathOperator{\N}{\mathbb{N}}
\DeclareMathOperator{\Z}{\mathbb{Z}}
\DeclareMathOperator{\Q}{\mathbb{Q}}
%%%%%%%%%%%%%%%%%%%%%%%%%%%%%%%%%%%%%%%%%%

\newenvironment{prova}
{\vspace{-0.2cm}\par\smallskip\noindent\textit{Demonstração.}\newline \indent}
{\vspace{-0.2cm}{\par\medskip\hfill$\square$\par}}



%%%%%%%%%%%%%%%% Organizção dos Teoremas e Proposições %%%%%%%%%%%%%%%%%%%%%%%%%%%

%% Exigencia do Claudemir

\newcounter{contador} % Contador personalizado
\NewDocumentEnvironment{resultado}{ O{} m }{
    \vspace{-0.2cm}
    \par\medskip % Espaçamento
    \ifnum\value{section}>1
      \setcounter{contador}{0}% Reinicia o contador no início de cada seção
    \fi
  \refstepcounter{contador}% Incrementa o contador
    \noindent\textbf{#2}\if\relax\detokenize{#1}\relax\else\hspace{0.1cm}\textbf{(#1)}\fi% Nome em negrito (obrigatório e opcional)
  \hspace{0.1cm}\textbf{\arabic{chapter}.\arabic{section}.\arabic{contador}} % Informações do contador
    \ignorespaces % Remove indentação
}{
    \par\medskip % Espaçamento
}

\newtheorem{myenvinner}{}[section]
\newenvironment{env}[1]{%
  \vspace{-0.3cm}
  \renewcommand\themyenvinner{{#1} \thesection.\arabic{myenvinner}}%
  \begin{myenvinner}%
  \normalfont
}{\vspace{-0.2cm}\end{myenvinner}%
}
