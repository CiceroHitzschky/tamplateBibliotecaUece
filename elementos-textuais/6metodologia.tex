\chapter{APLICAÇÕES NA ANÁLISE FUNCIONAL}
    \section{Natureza intrínseca da Análise Funcional}
    \section{Desigualdade de Hölder para integrais}
    \section{Desigualdade de Minkowski para integrais}
    \section{Desigualdade de Young}
    \section{Comentários sobre o Teorema de Regularidade de Schauder}
 A pesquisa que será realizada tem natureza básica, pois segundo Prodanov et al uma pesquisa básica
 \enquote{objetiva gerar conhecimentos novos úteis para o avanço da ciência sem aplicação prática prevista. Envolve verdades e interesses universais} (\citeyear{profreitas}, p.51). 

 Além disso, de acordo com Prodanov et al (\citeyear{profreitas}) a pesquisa também constará de uma abordagem quantitativa uma vez que os fatos podem ser relevados fora de uma complexidade social, econômico e político. Pode-se observar que a pesquisa terá objetivo exploratório, pois 

 \begin{citlon}
     As pesquisas exploratórias têm como propósito proporcionar maior familiaridade com o problema, com vistas a torná-lo mais explícito ou a construir hipóteses. 
     Seu planejamento tende a ser bastante flexível, pois interessa considerar os mais variados aspectos relativos ao tato ou fenômeno estudado.
     A coleta de dados pode ocorrer de diversas maneiras, mas geralmente envolve: 1.levantamento bibliográfico;
     2. entrevistas com pessoas que tiveram experiência prática com o assunto; 
     e 3. análise de exemplos que estimulem a compreensão (SELLTIZ et al., 1967, p. 63). 
     Em virtude dessa flexibilidade, torna-se difícil, na maioria dos casos, "rotular"os estudos exploratórios, mas é possível identificar pesquisas bibliográficas, estudos de caso e mesmo levantamentos de campo que podem ser considerados estudos exploratórios. (\citeauthor{gil}, \citeyear{gil}, p.33)
 \end{citlon}

 Como fora supracitado, pode-se atribuir vários procedimentos técnicos à pesquisas exploratórias. Dentre estes, o procedimento técnico será realizado por meio de uma pesquisa bibliográfica uma vez que esta será elaborada com base em material já publicado (\citeauthor{gil}, \citeyear{gil}) podendo alternar entre livros, teses e dissertações sobre o assunto.

 Diante disto, o texto será organizado em quatro capítulos sendo o primeiro destinado à uma contextualização histórica valorizando a importância e relevência do tema. 
 O segundo será composto da construção de requisitos para que possam ser apresentados adiante os espaços $L^p$ de maneira que o leitor não precise consultar outro texto para prosseguir. 
 Dentre tais, consta-se conceitos elementares de álgebra linear; cálculo diferencial e integral; e espaços métricos.
 O terceiro será constituído da apresentação dos espaços $L^p$ através de sua definição propriedades, proposições e teoremas importantes.

 Por fim, o último capítulo apresentará conceitos estendidos da álgebra linear trabalhando em espaços vetoriais normados que não são necessariamente finitos sendo este ambiente área de estudo da análise funcional onde serão realizadas as aplicações com a base exposta no capítulo três.